\documentclass[17pt]{beamer} %Makes presentation
%\documentclass[handout, 17pt]{beamer} %Makes Handouts
\usetheme{Singapore} %Gray with fade at top
\useoutertheme[subsection=false]{miniframes} %Supppress subsection in header
\useinnertheme{rectangles} %Itemize/Enumerate boxes
\usecolortheme{seagull} %Color theme
\usecolortheme{rose} %Inner color theme

\definecolor{light-gray}{gray}{0.75}
\definecolor{dark-gray}{gray}{0.55}
\setbeamercolor{item}{fg=light-gray}
\setbeamercolor{enumerate item}{fg=dark-gray}

\setbeamertemplate{navigation symbols}{}
%\setbeamertemplate{mini frames}[default]
%\setbeamercovered{dynamics}
\setbeamerfont*{title}{size=\Large,series=\bfseries}
\setbeamerfont{footnote}{size=\tiny}

%\setbeameroption{notes on second screen} %Dual-Screen Notes
%\setbeameroption{show only notes} %Notes Output

\setbeamertemplate{frametitle}{\vspace{.5em}\bfseries\insertframetitle}
\newcommand{\heading}[1]{\noindent \textbf{#1}\\ \vspace{1em}}

\usepackage{bbding,color,multirow,times,ccaption,tabularx,graphicx,verbatim,booktabs}
\usepackage{colortbl} %Table overlays
\usepackage[english]{babel}
%\usepackage[latin1]{inputenc}
%\usepackage[T1]{fontenc}
\usepackage{lmodern}

%\author[]{Thomas J. Leeper}
\institute[]{
  \inst{}%
  Department of Government\\London School of Economics and Political Science
}

\usepackage{tikz}
\usetikzlibrary{shapes,arrows,decorations.pathreplacing,calc}
\usepackage{multicol}

\title{Conclusions}

\date[]{}

\begin{document}

\frame{\titlepage}

\frame{\tableofcontents}

\section{Mixing methods?!}
\frame{\tableofcontents[currentsection]}

\frame{

\frametitle{What can be mixed?}

\begin{itemize}
\item Anything!
\vspace{0.5em}
\item Either:
	\begin{itemize}\footnotesize
	\item Methods that complement each other (trade-offs)
	\item Methods that can inform each other
	\end{itemize}
\item Common pairings
	\begin{itemize}\footnotesize
	\item Large-n quantitative and in-depth case studies
	\item Process-tracing methods with large-n or medium-n
	\item Interviews or archival work with anything else
	\item Observational and experimental methods
	\end{itemize}
\end{itemize}

}

\frame{

\frametitle{Mixing Methods: Why?}

\begin{itemize}\itemsep0.5em
\item<2-> All research is inadequate
\item<3-> Compensate for limitations of a given research design with the strengths of an alternative
\item<4-> Inform methods decisions with other (provisional) research
\item<5-> Strengthen a single causal claim with multiple forms of evidence
\end{itemize}
}

\frame{

\frametitle{Mixing Methods: How?}

\begin{itemize}\itemsep2em
\item<2-> Triangulation
	\begin{itemize}
	\item \textit{Conceptual} replication
	\item Accumulation
	\end{itemize}
\item<3-> Integration
	\begin{itemize}
	\item ``Synergy''
	\end{itemize}
\end{itemize}
}

	


\frame{

\frametitle{Triangulation}

\begin{itemize}\itemsep1em
\item Definition: approach the same research question, topic, or theory with different types of methods and/or data
\item Goal is \textit{replication}
	\begin{itemize}
	\item Do the inferences drawn from different research designs agree?
	\item<2-> How would we know if they agree \textit{enough}?
	\item<3-> What do we conclude from non-replication?
	\end{itemize}
\end{itemize}

}

\frame{

\frametitle{Triangulation: Example}

\begin{itemize}\itemsep1em
\item Brexit

\vspace{1em}

\item Large-\textit{n} analysis of survey data
\item Analysis of aggregated, district-level results
\item Qualitative interviews and/or focus groups
\end{itemize}

}


\frame{

\frametitle{Integration}

\begin{itemize}\itemsep1em
\item Definition: use one method to theorize or design a study using an additional method(s)
\item Goal is better research design
	\begin{itemize}
	\item<2-> What cases should we study? 
	\item<3-> What is a reasonable theory?
	\item<4-> How do we measure our concepts?
	\item<5-> What are plausible mechanisms?
	\item<6-> Have we missed any confounding factors?
	\end{itemize}
\end{itemize}
}

\frame{

\frametitle{Integration: Example 1}

\begin{itemize}\itemsep0.5em
\item Brexit: Qualitative driving quantitative

\vspace{1em}

\item Long-form qualitative interviews to identify how Britons think about Brexit
\item Large-n survey analysis that measures concepts identified in interviews
\end{itemize}

}


\frame{

\frametitle{Integration: Example 2}

\begin{itemize}\itemsep0.5em
\item Brexit: Quantitative driving qualitative

\vspace{1em}

\item Quantitative analysis regional voting patterns
\item In-depth case studies of:
	\begin{itemize}
	\item ``Typical'' cases
	\item ``Deviant'' cases
	\item ``Extreme'' cases
	\end{itemize}
\end{itemize}

}

\frame{
\centering
\includegraphics[height=\textheight]{images/brexit-exports}
}


\frame{

\frametitle{Why does this matter?}

\begin{itemize}\itemsep0.5em
\item<2-> Research is messy!
\item<3-> Most findings are probably false!
\item<4-> Contradictory findings drive new research!
	\begin{itemize}
	\item Scope conditions
	\item Heterogeneity
	\item Bad conceptualization
	\item Bad measurement
	\item Bad methods
	\item Bad inferences from evidence
	\end{itemize}
\end{itemize}

}


\frame{}

\section{Where do we go from here?}
\frame{\tableofcontents[currentsection]}

\frame{

\begin{itemize}
\item Content analysis, qualitative coding
\item Discourse analysis, framing analysis
\item Quantitative Comparative Analysis (QCA)
\item Focus groups, elite interviewing
\item Archival/historical evidence-gathering
\item Interpretative and post-positivist methods
\item Political theory
\end{itemize}

}

\frame{

\begin{itemize}
\item Factor analysis, principal components, IRT
\item Regression trees, classifiers, SVM
\item K-means clustering, hierarchical clustering
\item Nonparametric statistics
\item Bayesian statistics
\item Time series analysis and panel data
\item Quantitative text analysis
\item GIS, spatial data, mapping
\item ``Big data''
\end{itemize}

}


\frame{

\frametitle{Continuing Your Research}

\small

\begin{itemize}
\item LSE Groups\footnote{\url{https://info.lse.ac.uk/staff/divisions/Teaching-and-Learning-Centre/TLC-events-and-workshops/LSE-GROUPS}}
\item Dissertation (GV390)
\item Some research-based GV3xx courses
\item<2-> Online education\footnote{\url{https://www.coursera.org/}, \url{https://www.edx.org/}, \url{https://www.datacamp.com/}}
\item<2-> Postgraduate study
\end{itemize}

}

\frame{}

\section{Where have we been?}
\frame{\tableofcontents[currentsection]}

\frame{

\frametitle{Claims}

\begin{itemize}\itemsep1em

\item Politics is full of claims

\item The credibility of claims depends on the strength of evidence and argument

\item This class aims to give you tools to:

\begin{itemize}\itemsep1em
\item make credible claims, \textit{and}
\item evaluate claims made by others
\end{itemize}

\end{itemize}

}

\begin{frame}[fragile]

\frametitle{Drawing Inferences}

\vspace{-2em}

\begin{center}
\tikzstyle{block} = [rectangle, draw, fill=blue!20!white, text width=5em, text centered, rounded corners, minimum height=4em, node distance=7em]
\begin{tikzpicture}[scale=0.5]
\node<1-> [block] (claims) {Claim(s)};
\node<1-> [block, below of=claims] (beliefs) {Belief(s)};
\node<1-> [block, left of=beliefs] (filter) {Processing Filter};
\node<1-> [block, left of=filter] (evidence) {Evidence};

\draw<1-> [->, very thick] (evidence) -- (filter);
\draw<1-> [->, very thick] (filter) -- (beliefs);
\draw<1-> [->, dashed, very thick] (beliefs) -- (claims);

\draw<2-> [red, very thick] (-12,-8) ellipse (8cm and 4cm);
\draw<2-> (-12,-1) node (topic) {\Large\textcolor{red}{Focus of this class}};
\end{tikzpicture}
\end{center}

\end{frame}


\frame{\huge\vskip20pt\textbf{What have we learned since then?}}

% what is political science?
% what makes for interesting questions?
% how do we describe the world? and evaluate alternative ways of describing the world?
% how do we generate theories and accumulate knowledge?
% how do we draw causal inferences? what kinds of comparisons should we make?
% what are methods that can be used to generate causal inferences?
% how do we choose methods? how do we balance trade-offs between methods?


\frame{}


\frame{
	
\frametitle{The Exam!}

	\begin{center}
	\Large
	\textbf{What do you think will be on the exam?}
	\end{center}
	
}

\frame{

\frametitle{The Exam!}

The exam has three parts:

\begin{enumerate}
\item Short-answer questions
\item Essay analysing/evaluating an empirical article
\item Research proposal section
\end{enumerate}

Sample paper is on Moodle.

}


\begin{frame}

\frametitle{Part B Readings}

\footnotesize

%\begin{multicols}{2}
\begin{itemize}
\item Young and Soroka (2012) (MT7)
\item Goffman (2009) (MT8)
\item Campbell and Ross (1968) (MT11)
\item Tannenwald (1999) (LT3)
\item Lange, Mahoney, vom Hau (2006) (LT4)
\item Doner, Ritchie, Slater (2005) (LT4)
\item Hibbs (1978) (LT8)
\item Cusack, Iversen, Soskice (2007) (LT9)
\item Bhavnani (2009) (LT10)
\end{itemize}
%\end{multicols}

\end{frame}



\appendix
\frame{}

\end{document}
